\section{Introduction}
%Introduction
This assignment evaluates the Jaguar case according to the use of project management tools. Teradyne is at the time of the case a 45 year old semiconductor testing firm. The Jaguar project was about making an entirely new semiconductor test platform, where multiple types of devices could be tested.

The hardware development worked well, but the software development ran behind schedule. This ended up increasing the development costs by 35\%. Throughout the Jaguar project, a key customer, AlphaTech, considered going to a competitor of Teradyne for service. In the end AlphaTech decided to go with Teradyne anyways, but under the condition, that the project would be finished on time. The project was delivered on the agreed date, but the software did not incorporate all the requested features of AlphaTech. The software side of the project end up having to fix bugs for AlphaTech for 6 months. This delay impacted the further roll-out of the platform. Multiple factors had an impact on the delay. Prior to the project launch the company reorganized, folding into a single platform, which was a big change in structure. This assignment evaluates how Teradyne incorporated project managements tools and the impact it had on the Jaguar project.


\section{Position Statement}

%\emph{Introduction plus one to two paragraphs}

% Position Statement

Teradyne's use of project managements tools in the Jaguar project lacked the necessary buy-in and knowledge of team members. The project teams should have been trained more in using the project management tools, and how to derive value from these. In addition to this some project tools where used to hide, that the software side of the project was far behind schedule. The use of tools resulted in a transparrent project with a clear scope from the beginning, which left little room for experimentation or delays.

\section{Evaluation Criteria}

\begin{enumerate}
    %\item The advantages and disadvantages of the tools selected
    %\subitem Selecting the correct tools to use for managing the project. If the incorrect tools are selected for the task at hand, they can impede the progress of the project. 
    \item How the software development teams did or did not benefit from their usage of the project management tools.
    \subitem Accessing how the use of project management tools impacted the software development teams, in a positive or negative way.

    \item How the project was impacted by use of the project management tools.
    \subitem How the process and outcome of the jaguar project was impacted by the use the project management tools.
\end{enumerate}


\section{Proof of the evaluation}

\subsection{Software Teams' usage of software management tools}

The formalized project management tools that where used in the Jaguar case, was Work Breakdown Structure (WBS) \cite[p. 113]{Larson2021}, 3-points estimation, critical path analysis \cite[ch. 6]{Larson2021} and earned value analysis. These tools where selected, as O'Brien believed they would help the teams working on the Jaguar case, by giving them additional data and details on the progress of tasks to the team. \cite[p. 7]{GinoPisano2005}. 
%And as the case would show, additional data and details where provided. However, with the teams not being properly trained in using these tools,, they would end up giving more problems than gain. 
As George Conner, the person responsible for the architecture on the Jaguar project, stated. The tools would eventually create too much data for the teams to handle. They didn't know how to properly process the data, which resulted in them not being able to see the entire picture behind the overwhelming amount of data.\cite[p. 13]{GinoPisano2005}
Another sign of the teams not having enough knowledge of how to use the tools, became present when the first year of the program had passed, and various project metrics indicated that software was running at approximately 50\% earned value per month. The program manager Kevin Giebel's take on this, was because the management team had a skepticism around the metric, and they, because of this, would not pay enough attention to the data presented to them \cite[p. 9]{GinoPisano2005}. Which might could have been avoided by teaching the management to have more confidence in this metric. 

It is quite clear that the tools did not bring the desired value to the project, and they might even have done more damage than good.
The engineering manager Ben Brown's take on why the tools went so wrong was that the people would become more concerned about the metric in itself and not about what a poor metric was telling them. It is important not to make the metric become the goal, such that the project does not shift its focus to satisfy the metric. He states that if the tools where accepted and understood by the team, these challenges would be solved \cite[p. 12]{GinoPisano2005}.
Looking at the reflection on the case, it is clear that the teams where not comfortable with using these tools. Giebel commented that the team members often did not know how to gain value from the tools, and where simply just using them because of an obligation to do so. He also states, that with more experience, and perhaps some additional training, this problem would be rectified \cite[p. 11]{GinoPisano2005}.
O'Brien's point of view on this, was that the team was unable to react to the data that was presented to them \cite[p. 11]{GinoPisano2005}.
Carbone's take on it was that the tools where simply used wrong. The software team lied to themselves by using them. Instead of getting to a point where a delay of the critical path would show, where additional resources should be allocated, the team would rejigger the critical path. So instead of facing the problems the critical path presented for them, they would put non-parallel tasks in parallel to save time, and add resources to the critical path, to make it fit. Even though the earned value analysis would actually show the software disaster, this was ignored. \cite[p. 11]{GinoPisano2005}

\begin{comment}
    Pros:
Page 1:
"With the data and information provided by the new tools we where using, we where able to know whether a team was kidding itself or was having work done at the right pace."

Page 8:
Overall view on software
"In software, you don't have these physical constraints. You can generally do tasks in almost any order. This gives you a lot more flexibility (as you execute) to shuffle people around to different tasks, and to even change the order of the tasks."

Page 7
"O'Brien was a strong believer in the value of these tools, particularly for a complex project like Jaguar"

Page 8
"O'Brien was convinced that these methodologies would provide a robust means to communicate the project status to management, and to identify critical issues (such as potential delays) that required senior management's action or support."
"... Giebel was also responsible for "keeping the team honest.""

Cons:
Page 1:
End result for the Jaguar project
"Yet, at the same time, software, a major component of the program, had run badly behind schedule and was still not completed."

Page 5:
APP
"Some people where going through the motions of using the tools but wheren't really changing their behaviors. They were still over-committing. They where still coming up with unrealistic schedules. They wheren't being intellectually honest."

Page 9:
Complications with new OS
"While the hardware subsystems where largely able to keep on track, software development emerged as a problem. (See exhibit 7).
The new platform would utilize a Windows NT-based operating system called IG-XL that had been developed at Teradyne's Boston site for use on the platform called FLEX. Because Boston's software group was busy developing extensions to the existing FLEX product line and fixing bugs, the development of the Jaguar software had to be staffed primarily from Agoura."

Page 9:
Complications with allocating experts to fitting teams
"Most of the developers had never worked with IG-XL before. A few had limited familiarity with an older generation of IG-XL. The experts on the IG-XL platform were located in Boston and were focused on extending and fixing the FLEX code base. These experts had little time to spend on Jaguar development. At that time the company priority was on FLEX, with frequent statements to the effect that, "If FLEX doesn't succeed, there won't be any market for Jaguar." The Jaguar development team underestimated the extent of the learning curve on the new platform. Even with what were intended and believed to be conservative estimates, we were running late."

Page 9:
Not paying enough attention to the data
"Various project metrics indicated a problem. For instance, for the first year of the program, software was running at approximately 50\% earned value per month. If this were correct, this meant that software completing only one-half of the tasks that they had originally planned. Kevin Giebel noted: "Software was in denial. They kept saying they would catch up." When asked why the core team management or senior management did not pay enough attention, Giebel reflected, "One of the reasons was that the management team did not pay enough attention to the data because of its skepticism around the metric." Conner added, "The software's problem emerged gradually. We just didn't see it until very late, but we all knew it was screwed up."

Page 10:
Trying to catch up
"... the software fell further behind schedule. In January 2004, senior management committed 15 additional software engineers to the project to counter the problem."

Page 10:
How stressed the software team was
"As the deadline closed in, the software team shifted its effort almost completely to fixing bugs and getting an acceptable, operational piece of software to AlphaTech. Carbone stated: "The software team was under enormous pressure, and it just kept getting worse as it was slipping. The stress levels were off the charts. There was a lot of burnout. The fact that the lost very few people along the way is a tribute to Paul [Roush's] leadership. That team was incredibly loyal to Paul."

Page 10:
Trouble with software development at Teradyne
""When you work with hardware there are fixed gates in the process: the first board, the artwork, etc. These tangible, hard points in the process. If they are not done, you know it. You can't lie to yourself. With software, it's much squishier. You don't have these points." Conner thought the problem were much more endemic to the company: "At Teradyne, we have an intuitive feel for what the problems in hardware are. We don't have that feel in software.""

Page 10-11:
Teradyne's shipping the first complete system to AlphaTech
"All of the hardware subsystems met their specifications. The software did not incorportate all of the features initially requested by AlphaTech, and it was laden with bugs, but it was functional. For the next six months, the software team focused solely on upgrading the software and fixing bugs for AlphaTech. Roush noted, "We basically stopped doing development at that point, and just worked on bugs." And cabone recalled, "They had to shift to pure firefighting mode. Any sense of process went out the window. They were no longer doing development; they were just trying to fix bugs for AlphaTech. By the end, they were coding day-by-day and uploading the software to AlphaTech over the Web." In June 2004, additional software engineers were added to the project."

Page 11:
"The victory, however, came with a cost."
"Much of the rest of the project-including development of features for other customer-was delayed. In addition, software, completely consumed with bug fixing, fell further behind schedule. Additional software engineers were once again added to the project. In July 2004, Carbone was appointed to lead the remaining software development. "The situation was a mess. The people were burnt out. We had to add 50 more developers. We just used brute force. It wasn't pretty. We're still digging out.""

Page 11
"Ultra Flex" delays
"However, due to delays in getting the software online, the volume production ramp of the product was pushed out six months."

Page 11:
Reflections on the tools
"Giebel commented, "Too often, team members didn't know 'how to get value from the tools they were using' and thought they 'could have figured out what was wrong without them.'" Giebel believed that with more experience, and perhabs some additional traning, this problem would be rectified. 

Page 11: 
Reflection on the software problem
"In recalling the struggles of the software team, Carbone noted: "The tools allowed the software team to lie to themselves. They kept rejiggering the critical path, putting things in parallel, adding resources, etc., to make it fit. Some very strong people allowed themselves to be fooled by the data. Jack let the metrics lie to him. The software disaster was evident from the EV. but we ignored it (see Exhibit 9)."

Page 12:
Concern about the metric
"It was natural that over time some people became more concerned about the metric in itself and not about what a poor metric was telling them. Plus, anyone can make any metric look good. You have to be careful: the metric might become the goal, so people focus on managing the metric rather than the project. People fall into this trap not because they want to do the wrong things but because they feel pressure to manage to the metric. People need tools but, more importantly, they need the attitude. I do not think more sophisticated tools are necessarily better. Tools make things better if people using them accept and understand what they are for and how they work."

Page 12:
Primavera
"Primarevera requires a very static work breakdown strucutre; once you enter it, it is very difficult to modify. The problem is that as you execute a project like this, you actually discover things you have to do differently. But, the schedule is produced and updated using the original work breakdown structure. So the reported schedules becomes less meaningful over time." 

Page 13:
Information overload
"People were focusing on details so much that they could not see the entire picture. In general, the tools do not help you focus on what decisions to make, they only provide you data and details on the progress of tasks. And the amount of data they provide can be overwhelming."
\end{comment}

\subsection{Usage of software management tools impact on the project}
\paragraph{Clear scope upfront} Teradyne was used to go all-in on front-end-sizing and defeature later, as to hit the deadline. On the Jaguar project, they took the opposite approach to hit the market window.  This was characterized as an uncomfortable change. \cite[p. 8]{GinoPisano2005} This made led to a more strict development process, where features should be clearly defined beforehand, which lead to less room for experimentation.

\paragraph{Focus on meeting deadline} The team used a lot of time on analyzing the critical path \cite[p. 8]{GinoPisano2005}, therefore the progress of the project was always visible to all members. This made the workers more afraid to commit changes. They also changed their committed ship date from June 30 2004 to March 31 2004, even though June 30 was the projected deadline. \cite[p. 12]{GinoPisano2005} This date was used in the CP report and kept lots of tension in the early milestones.

They where squeezing the schedule as tight as they could without making the necessary changes to boost the productivity to actually reach that goal. The use of tools to track a process is good to keep track of a project, but when the schedule is tight it can stress workers and in the end do more harm than good. 

\paragraph{Transparency of the project} Some liked the fact that everything was visible: "The tools provided visibility into the project that we never used to have" \cite[p. 12]{GinoPisano2005}
Transparency of a project can both be positive and negative, because everyone can see the progress and it can be a motivation factor. Although it can have the opposite effect too. It can be a demotivating when things do not go as planned. Especially when you fall very far behind it can seem like a marathon to get back on track. Also being reminded that you are behind is often irrelevant because it might be too late to fix. "The tools tell you if you're late, but you shouldn't need a tool to tell you that. If you have to find out from the tool, it's already too late" \cite[p. 12]{GinoPisano2005}

\paragraph{Loss in sense of responsibility} The use of tools made some feel less responsible of the project: "I didn't feel responsible for the success of the entire project. I felt responsible for reporting data" \cite[p. 12]{GinoPisano2005}
This is a negative aspect of using the tools. The tools are made to bring value for the project and make smoother development project. Instead it some felt they where using the tools for the sake of using them, but didn't do any good for them. This might have contributed in delaying the project since the time used on the tools could instead be used on actual work for the project.

\section{Qualifications}

There are other factors present in this case, that also could have affected the outcome of the project.
One of the factors where the technical difficulties Teradyne faced choosing the operating system the new platform should utilize. IG-XL was chosen, and had been developed at Teradyne's Boston site for the use on a platform called FLEX. As the Boston people where busy further improving this, the only developers left to work on this where the ones from Agoura, who had little to no experience with IG-XL. Paul Roush also states: "The Jaguar development team underestimated the extent of the learning curve on the new platform."\cite[p. 9]{GinoPisano2005}. This obviously also affected the outcome of the project.

Another factor was regarding the lack of time the team was presented with by taking the AlphaTech case, and having to push their release of their product 3 months earlier. It created a huge motivation for the team members, by having a concrete deadline and a "real customer" \cite[p. 10]{GinoPisano2005}. However, pushing the deadline 3 months earlier created an enormous pressure on the team members to both work efficiently and work more. This caused a high level of stress and a lot of burnouts, which might have resulted in the team members not working as efficient as usual \cite[p. 10]{GinoPisano2005}. 

Looking at Exhibit 6 \cite[p. 20]{GinoPisano2005}, it it also very clear that Teradyne is a company with more focus on the hardware. As Conner also stated: "At Teradyne, we have an intuitive feel for what the problems in hardware are. We don't have that feel in software."\cite[p. 10]{GinoPisano2005} The lack of software development experience could have affected the outcome for the software development team. 
