\section{Position Statement}

%Introduction
This assignment evaluates the Jaguar case according to the use of project management tools. Teradyne is a, at that time, 45 year old semiconductor firm. Teradyne launched the jaguar project about making an entirely new semiconductor test-system platform. The hardware development worked well, but the software development ran behind schedule. This ended up increasing the development cost's by 35\%. Throughout the jaguar project a key customer, Alpha Tech, where looking change from Teradyne to a competitor. In the end they decided to stay with Teradyne, but under the circumstances, that the project wouldn't be delayed. The project ended up being delayed about a year, which was far from the goal. Multiple factors had an impact on the delay. This assignment looks at how Teradyne incorporated project managements tools in the jaguar project and the impact it had.

%\emph{Introduction plus one to two paragraphs}

% Position Statement

Teradyne's use of project managements tools in the jaguar project lacked the necessary buy-in of some team members. In addition to this some project tools were used to hide, that the software side of the project was far behind schedule.

\section{Evaluation Criteria}

\begin{enumerate}
    %\item The advantages and disadvantages of the tools selected
    %\subitem Selecting the correct tools to use for managing the project. If the incorrect tools are selected for the task at hand, they can impede the progress of the project. 
    \item How the software development teams did or did not benefit from their usage of the project management tools.
    \subitem Accessing how the use of project management tools impacted the software development teams, in a positive or negative way.
    
    \item How the Jaguar project was impacted by the teams' usage of the project management tools.
    \subitem How the process and outcome of the project was impacted by how the teams' used the project management tools.
\end{enumerate}


\section{Proof of the evaluation}

\subsection{Software Teams}

Pros:
Page 8:
Overall view on software
"In software, you don't have these physical constraints. You can generally do tasks in almost any order. This gives you a lot more flexibility (as you execute) to shuffle people around to different tasks, and to even change the order of the tasks."

Cons:
Page 1:
End result for the Jaguar project
"Yet, at the same time, software, a major component of the program, had run badly behind schedule and was still not completed."

Page 9:
Complications with new OS
"While the hardware subsystems were largely able to keep on track, software development emerged as a problem. (See exhibit 7).
The new platform would utilize a Windows NT-based operating system called IG-XL that had been developed at Teradyne's Boston site for use on the platform called FLEX. Because Boston's software group was busy developing extensions to the existing FLEX product line and fixing bugs, the development of the Jaguar software had to be staffed primarily from Agoura."

Page 9:
Complications with allocating experts to fitting teams
"Most of the developers had never worked with IG-XL before. A few had limited familiarity with an older generation of IG-XL. The experts on the IG-XL platform were located in Boston and were focused on extending and fixing the FLEX code base. These experts had little time to spend on Jaguar development. At that time the company priority was on FLEX, with frequent statements to the effect that, "If FLEX doesn't succeed, there won't be any market for Jaguar." The Jaguar development team underestimated the extent of the learning curve on the new platform. Even with what were intended and believed to be conservative estimates, we were running late."

Page 9:
Not paying enough attention to the data
"Various project metrics indicated a problem. For instance, for the first year of the program, software was running at approximately 50\% earned value per month. If this were correct, this meant that software completing only one-half of the tasks that they had originally planned. Kevin Giebel noted: "Software was in denial. They kept saying they would catch up." When asked why the core team management or senior management did not pay enough attention, Giebel reflected, "One of the reasons was that the management team did not pay enough attention to the data because of its skepticism around the metric." Conner added, "The software's problem emerged gradually. We just didn't see it until very late, but we all knew it was screwed up."

Page 10:
Trying to catch up
"... the software fell further behind schedule. In January 2004, senior management committed 15 additional software engineers to the project to counter the problem."

Page 10:
How stressed the software team was
"As the deadline closed in, the software team shifted its effort almost completely to fixing bugs and getting an acceptable, operational piece of software to AlphaTech. Carbone stated: "The software team was under enormous pressure, and it just kept getting worse as it was slipping. The stress levels were off the charts. There was a lot of burnout. The fact that the lost very few people along the way is a tribute to Paul [Roush's] leadership. That team was incredibly loyal to Paul."

Page 10:
Trouble with software development at Teradyne
""When you work with hardware there are fixed gates in the proccess: the first board, the artwork, etc. These tangible, hard points in the process. If they are not done, you know it. You can't lie to yourself. With software, it's much squishier. You don't have these points." Conner thought the problem were much more endemic to the company: "At Teradyne, we have an intuitive feel for what the problems in hardware are. We don't have that feel in software.""

Page 10-11:
Teradyne's shipping the first complete system to AlphaTech
"All of the hardware subsystems met their specifications. The software did not incorportate all of the features initially requested by AlphaTech, and it was laden with bugs, but it was functional. For the next six months, the software team focused solely on upgrading the software and fixing bugs for AlphaTech. Roush noted, "We basically stopped doing development at that point, and just worked on bugs." And cabone recalled, "They had to shift to pure firefighting mode. Any sense of process went out the window. They were no longer doing development; they were just trying to fix bugs for AplhaTech. By the end, they were coding day-by-day and uploading the software to AlphaTech over the Web." In June 2004, additional software engineers were added to the project."

Page 11:
"The victory, however, came with a cost."
"Much of the rest of the project-including development of features for other customer-was delayed. In addition, software, completely consumed with bug fixing, fell further behind schedule. Additional software engineers were once again added to the project. In July 2004, Carbone was appointed to lead the remaining software development. "The situation was a mess. The people were burnt out. We had to add 50 more developers. We just used brute force. It wasn't pretty. We're still digging out.""

Page 11
"Ultra Flex" delays
"However, due to delays in getting the software online, the volume production ramp of the product was pushed out six months."

Page 11:
Reflections on the tools
"Giebel commented, "Too often, team members didn't know 'how to get value from the tools they were using' and thought they 'could have figured out what was wrong without them.'" Giebel believed that with more experience, and perhabs some additional traning, this problem would be rectified. 

Page 11: 
Reflection on the software problem
"In recalling the struggles of the software team, Carbone noted: "The tools allowed the software team to lie to themselves. They kept rejiggering the critical path, putting things in parallel, adding resources, etc., to make it fit. Some very strong people allowed themselves to be fooled by the data. Jack let the metrics lie to him. The software disaster was evident from the EV. but we ignored it (see Exhibit 9)."

Page 12:
Concern about the metric
"It was natural that over time some people became more concerned about the metric in itself and not about what a poor metric was telling them. Plus, anyone can make any metric lookgood. You have to be careful: the metric might become the goal, so people focus on managing the metric rather than the project. People fall into this trap not because they want to do the wrong things but because they feel pressure to manage to the metric. People need tools but, more importantly, they need the attitude. I do not think more sophisticated tools are necessarily better. Tools make things better if people using them accept and understand what they are for and how they work."

Page 12:
Primavera
"Primarevera requires a very static work breakdown strucutre; once you enter it, it is very difficult to modify. The problem is that as you execute a project like this, you actually discover things you have to do differently. But, the schedule is produced and updated using the original work breakdown structure. So the reported schedules becomes less meaningful over time." 

\subsection{The Project as a Whole}

\emph{App.\ two paragraphs per criteria}

\section{Qualifications}

% Lack of resources to begin with

\emph{Max half a page}

\nocite{Larson2021}