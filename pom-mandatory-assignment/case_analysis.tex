\section{Introduction}
%Introduction
This assignment evaluates the Jaguar case according to the use of project management tools. Teradyne is at the time of the case a 45 year old semiconductor testing firm. The Jaguar project was about making an entirely new semiconductor test platform, where multiple types of devices could be tested.

The hardware development worked well, but the software development ran behind schedule. This ended up increasing the development costs by 35\%. Throughout the Jaguar project, a key customer, AlphaTech, considered going to a competitor of Teradyne for service. In the end AlphaTech decided to go with Teradyne anyways, but under the condition, that the project would be finished on time. The project was delivered on the agreed date, but the software did not incorporate all the requested features of AlphaTech. The software side of the project end up having to fix bugs for AlphaTech for 6 months. This delay impacted the further roll-out of the platform. Multiple factors had an impact on the delay. Prior to the project launch the company reorganized, folding into a single platform, which was a big change in structure. This assignment evaluates how Teradyne incorporated project managements tools and the impact it had on the Jaguar project.


\section{Position Statement}

%\emph{Introduction plus one to two paragraphs}

% Position Statement

Teradyne's use of project managements tools in the Jaguar project lacked the necessary buy-in and knowledge of team members. The project teams should have been trained more in using the project management tools, and how to derive value from these. In addition to this some project tools were used to hide, that the software side of the project was far behind schedule.



\section{Evaluation Criteria}

\begin{enumerate}
    %\item The advantages and disadvantages of the tools selected
    %\subitem Selecting the correct tools to use for managing the project. If the incorrect tools are selected for the task at hand, they can impede the progress of the project. 
    \item How the software development teams did or did not benefit from their usage of the project management tools.
    \subitem Accessing how the use of project management tools impacted the software development teams, in a positive or negative way.

    \item How the project was impacted by use of the project management tools.
    \subitem How the process and outcome of the jaguar project was impacted by the use the project management tools.
\end{enumerate}


\section{Proof of the evaluation}

\subsection{Software Teams' usage of software management tools}

The formalized project management tools that were used in the Jaguar case, were Work Breakdown Structure (WBS) \cite[p. 113]{Larson2021}, 3-points estimation, critical path analysis \cite[ch. 6]{Larson2021} and earned value analysis. These tools were selected, as O'Brien believed they would help the teams working on the Jaguar case, by giving them additional data and details on the progress of tasks to the team. \cite[p. 7]{GinoPisano2005}. 
%And as the case would show, additional data and details were provided. However, with the teams not being properly trained in using these tools,, they would end up giving more problems than gain. 
As George Conner, the person responsible for the architecture on the Jaguar project, stated. The tools would eventually create too much data for the teams to handle. They didn't know how to properly process the data, which resulted in them not being able to see the entire picture behind the overwhelming amount of data.\cite[p. 13]{GinoPisano2005}
Another sign of the teams not having enough knowledge of how to use the tools, became present when the first year of the program had passed, and various project metrics indicated that software was running at approximately 50\% earned value per month. The program manager Kevin Giebel's take on this, was because the management team had a skepticism around the metric, and they, because of this, would not pay enough attention to the data presented to them \cite[p. 9]{GinoPisano2005}. Which might could have been avoided by teaching the management to have more confidence in this metric. 

It is quite clear that the tools did not bring the desired value to the project, and they might even have done more damage than good.
The engineering manager Ben Brown's take on why the tools went so wrong was that the people would become more concerned about the metric in itself and not about what a poor metric was telling them. It is important not to make the metric become the goal, such that the project does not shift its focus to satisfy the metric. He states that if the tools were accepted and understood by the team, these challenges would be solved \cite[p. 12]{GinoPisano2005}.
Looking at the reflection on the case, it is clear that the teams were not comfortable with using these tools. Giebel commented that the team members often did not know how to gain value from the tools, and were simply just using them because of an obligation to do so. He also states, that with more experience, and perhaps some additional training, this problem would be rectified \cite[p. 11]{GinoPisano2005}.
O'Brien's point of view on this, was that the team was unable to react to the data that was presented to them \cite[p. 11]{GinoPisano2005}.
Carbone's take on it was that the tools were simply used wrong. The software team simply lied to themselves by using them. Instead of getting to a point of where a delay of the critical path would show where additional resources should be allocated, the team would rejigger the critical path. So instead of facing the problems the critical path should present for them, they would put non-parallel tasks in parallel to save time, and add resources to the critical path, to make it fit. And even though the earned value analysis would actually show the software disaster, this was ignored. \cite[p. 11]{GinoPisano2005}


\subsection{Usage of software management tools impact on the project}
\textbf {Clear scope of the project}
\\
Teradyne was used to go all-in on front-end-sizing and defeature later but this time they did the opposite, which was uncomfortable for them "In the past we tended to go "all in" on front-end sizing, and we defeatured the system later when we couldn't hit the schedule. On the jaguar, we had to take the opposite approach to be sure we would hit the market window." This was an uncomfortable change" \cite[p. 8, ]{GinoPisano2005} This made it a more strict development process, where features should be clearly defined beforehand, which lead to less room for experimentation.
\\
\textbf {Focus on meeting deadline}
\\
The team used a lot of time on analyzing the CP\cite[p. 8]{GinoPisano2005}, therefore the progress of the project was always visible to all members. This made the workers more afraid to commit changes. They also changed their committed ship date from June 30 2004 to March 31 2004, eventhough June 30 was the projected deadline. \cite[p. 12]{GinoPisano2005} This date was used in the CP report and kept lots of tension in the early milestones.
They where squeezing the schedule as tight as they could without making the necessary changes to boost the productivity to actually reach that goal. The use of tools to track a process is good to keep track of a project, but when the schedule is tight it can stress workers more. 
\\
\textbf {Transparency of the project}
\\
Some liked the fact that everything was visible: "The tools provided visibility into the project that we never used to have" \cite[p. 12]{GinoPisano2005}
Transparrency of a project can both be positive and negative, because everyone can see the progress and it can be a motivation factor. Although it can have the opposite effect too. It can be a unmotivating when things dosen't go as planned. Especially when you fall very far behind it can seem like a marathon to get back on track. Also being reminded that you are behind is often irrelevant because it might be too late to fix."The tools tell you if you're late, but you shouldn't need a tool to tell you that. If you have to find out from the tool, it's already too late" \cite[p. 12]{GinoPisano2005}
\\
\textbf {Loss in sense of responsibility}
\\
The use of tools made some feel less responsible of the project: "I didn't feel responsible for the success of the entire project. I felt responsible for reporting data" \cite[p. 12]{GinoPisano2005}
This is a negative aspect of using the tools. The tools are made to bring value for the project and make smoother development project. Instead it some felt they where using the tools for the sake of using them, but didn't do any good for them. This might have contributed in delaying the project since the time used on the tools could instead be used on actual work for the project.

\section{Qualifications}

There are other factors present in this case, that also could have affected the outcome of the project.
One of the factors were the technical difficulties Teradyne faced when choosing which operating system the new platform should utilize. IG-XL was chosen, and had been developed at Teradyne's Boston site for use on the platform called FLEX. As the Boston people were busy further improving this, the only developers left to work on this were the ones from Agoura, who had little to no experience with IG-XL. Paul Roush also states: "The Jaguar development team underestimated the extent of the learning curve on the new platform."\cite[p. 9]{GinoPisano2005}. This obviously also affected the outcome of the project.

Another factor was regarding the lack of time the team was presented with by taking the AlphaTech case, and having to push their release of their product 3 months earlier. This caused multiple thing. It created a huge motivation for the team members, by having a concrete deadline and a "real customer" \cite[p. 10]{GinoPisano2005}. However, pushing the deadline 3 months earlier created a enormous pressure on the team members to work efficient and a lot. This caused very high level of stress and a lot of burnouts, which could have resulted in the team members not working as efficient as usual \cite[p. 10]{GinoPisano2005}. 

Looking at Exhibit 6 \cite[p. 20]{GinoPisano2005}, it it also very clear that Teradyne is a company with more focus on the hardware. As Conner also stated: "At Teradyne, we have an intuitive feel for what the problems in hardware are. We don't have that feel in software."\cite[p. 10]{GinoPisano2005} The lack of software development experience could have affected the outcome for the software development team. 
